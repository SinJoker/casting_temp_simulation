\documentclass[UTF8]{ctexart}
\usepackage[T1]{fontenc}
\usepackage{fontspec}
\usepackage{lmodern}
\usepackage[a4paper, margin=1in]{geometry}
\usepackage{amsmath}
\usepackage{blkarray}
\usepackage{tikz}
\usepackage{graphicx}
\usepackage{booktabs}
\usepackage{array}
\usetikzlibrary{matrix, positioning}
\usepackage{hyperref}

\title{二维热传导方程离散的显式五对角矩阵结构}
\author{数值传热学分析}
\date{\today}

\begin{document}

\maketitle

\section{引言}
本文展示二维热传导方程隐式离散格式如何直接导出五对角矩阵。考虑 $3 \times 3$ 网格系统(总自由度 $N=9$),通过行优先排序直接构造完整矩阵。

\section{矩阵构造原理}

\subsection{离散方程}
中心节点 $(i,j)$ 的离散方程:
$$
\begin{split}
- r_x T_{i-1,j} &- r_y T_{i,j-1} + (1 + 2r_x + 2r_y) T_{i,j} \\
&- r_x T_{i+1,j} - r_y T_{i,j+1} = T_{i,j}^n
\end{split}
$$

\subsection{行优先索引}
网格点 $(i,j)$ 的一维索引:
$$
k(i,j) = i \times N_y + j \quad (N_y = 3)
$$

\begin{table}[h]
\centering
\caption{网格点 $(i,j)$ 到一维索引 $k$ 的映射}
\label{tab:index_mapping}
\begin{tabular}{c|c|c}
$i$ & $j$ & $k$ \\ \hline
0 & 0 & 0 \\
0 & 1 & 1 \\
0 & 2 & 2 \\
1 & 0 & 3 \\
1 & 1 & 4 \\
1 & 2 & 5 \\
2 & 0 & 6 \\
2 & 1 & 7 \\
2 & 2 & 8 \\
\end{tabular}
\end{table}

\section{完整五对角矩阵}
矩阵元素 $A_{k,l}$ 表示行 $k$ 和列 $l$ 的耦合系数:

$$
A = \scalebox{0.85}{$
\begin{blockarray}{ccccccccc}
 & k=0 & k=1 & k=2 & k=3 & k=4 & k=5 & k=6 & k=7 & k=8 \\
\begin{block}{c(ccccccccc)}
k=0 & d  & -r_y & 0    & -r_x & 0    & 0    & 0    & 0    & 0    \\
k=1 & -r_y & d  & -r_y & 0    & -r_x & 0    & 0    & 0    & 0    \\
k=2 & 0    & -r_y & d  & 0    & 0    & -r_x & 0    & 0    & 0    \\
k=3 & -r_x & 0    & 0    & d  & -r_y & 0    & -r_x & 0    & 0    \\
k=4 & 0    & -r_x & 0    & -r_y & d  & -r_y & 0    & -r_x & 0    \\
k=5 & 0    & 0    & -r_x & 0    & -r_y & d  & 0    & 0    & -r_x \\
k=6 & 0    & 0    & 0    & -r_x & 0    & 0    & d  & -r_y & 0    \\
k=7 & 0    & 0    & 0    & 0    & -r_x & 0    & -r_y & d  & -r_y \\
k=8 & 0    & 0    & 0    & 0    & 0    & -r_x & 0    & -r_y & d  \\
\end{block}
\end{blockarray}
$}
$$
其中 $d = 1 + 2r_x + 2r_y$。

\section{矩阵结构可视化}
\label{sec:matrix_visualization}

\begin{tikzpicture}[
    matrix of math nodes,
    nodes in empty cells,
    every node/.style={minimum size=1cm, anchor=center},
    row 1/.style={nodes={draw=none}},
    column 1/.style={nodes={draw=none}},
    band/.style={fill=gray!30}
]

\matrix[
    matrix of math nodes,
    nodes={text width=1cm, text height=1cm, align=center},
    row sep=-\pgflinewidth,
    column sep=-\pgflinewidth,
    nodes in empty cells
] (M) {
d & -r_y & 0 & -r_x & 0 & 0 & 0 & 0 & 0 \\
-r_y & d & -r_y & 0 & -r_x & 0 & 0 & 0 & 0 \\
0 & -r_y & d & 0 & 0 & -r_x & 0 & 0 & 0 \\
-r_x & 0 & 0 & d & -r_y & 0 & -r_x & 0 & 0 \\
0 & -r_x & 0 & -r_y & d & -r_y & 0 & -r_x & 0 \\
0 & 0 & -r_x & 0 & -r_y & d & 0 & 0 & -r_x \\
0 & 0 & 0 & -r_x & 0 & 0 & d & -r_y & 0 \\
0 & 0 & 0 & 0 & -r_x & 0 & -r_y & d & -r_y \\
0 & 0 & 0 & 0 & 0 & -r_x & 0 & -r_y & d \\
};

% 标记不同对角线
\draw[red, very thick] (M-1-1) -- (M-9-9);
\node[above=0.5cm of M-1-1, red] {主对角线 (d)};

\draw[blue, very thick] (M-1-2) -- (M-8-9);
\node[above=0.3cm of M-1-2, blue] {上对角线1 (-r_y)};

\draw[green, very thick] (M-1-4) -- (M-6-9);
\node[above=0.8cm of M-1-4, green] {上对角线Ny (-r_x)};

\draw[orange, very thick] (M-2-1) -- (M-9-8);
\node[below=0.5cm of M-9-8, orange] {下对角线1 (-r_y)};

\draw[purple, very thick] (M-4-1) -- (M-9-6);
\node[below=0.8cm of M-9-6, purple] {下对角线Ny (-r_x)};

% 标记矩阵块
\draw[thick, dashed] (M-3-3) -- (M-3-6);
\draw[thick, dashed] (M-3-3) -- (M-6-3);
\draw[thick, dashed] (M-3-6) -- (M-6-6);
\draw[thick, dashed] (M-6-3) -- (M-6-6);

\node[above=0.2cm of M-1-1] {行0};
\node[above=0.2cm of M-1-4] {行1};
\node[above=0.2cm of M-1-7] {行2};

\node[left=0.2cm of M-1-1, rotate=90] {列0};
\node[left=0.2cm of M-4-1, rotate=90] {列1};
\node[left=0.2cm of M-7-1, rotate=90] {列2};

\end{tikzpicture}

\section{矩阵元素结构总结}
\label{sec:matrix_structure}

矩阵包含五条带:
\begin{enumerate}
\item 主对角线(红色):所有 $A_{k,k} = d = 1 + 2r_x + 2r_y$
\item 上对角线1(蓝色):$A_{k,k+1} = -r_y$(行内右邻居)
\item 下对角线1(橙色):$A_{k,k-1} = -r_y$(行内左邻居)
\item 上对角线Ny(绿色):$A_{k,k+Ny} = -r_x$(下行同列邻居)
\item 下对角线Ny(紫色):$A_{k,k-Ny} = -r_x$(上行同列邻居)
\end{enumerate}

\section{边界条件处理}
\label{sec:boundary_conditions}

\subsection{边界缺失元素}
对于边界点,缺失的元素结构:
\begin{itemize}
\item 左边界($j=0$):下对角线1元素缺失
\item 右边界($j=Ny-1$):上对角线1元素缺失
\item 上边界($i=0$):下对角线Ny元素缺失
\item 下边界($i=Nx-1$):上对角线Ny元素缺失
\end{itemize}

\subsection{第二类边界条件}
例如左边界($i=0, j=0$)的Neumann条件:
$$
\begin{bmatrix}
d' & -r_y & 0 & \cdots \\
-r_y & d & \ddots & \\
0 & \ddots & \ddots & \\
\vdots & & & \\
\end{bmatrix}
$$
其中 $d'$ 由边界热流确定。

\section{结论}
二维热传导方程隐式离散直接形成五对角矩阵:
\begin{itemize}
\item 五条对角线对应五个非零元素带
\item 主对角线:自身系数 $1 + 2r_x + 2r_y$
\item 偏移$\pm 1$:行内邻居耦合 $(-r_y)$
\item 偏移$\pm N_y$:行间邻居耦合 $(-r_x)$
\end{itemize}
此结构揭示了热传导的局部耦合特性,为高效数值求解奠定基础。

\end{document}