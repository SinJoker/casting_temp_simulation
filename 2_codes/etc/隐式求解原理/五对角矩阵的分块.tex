\documentclass{ctexart}
\usepackage[a4paper, margin=1in]{geometry}
\usepackage{amsmath}
\usepackage{amssymb}
\usepackage{booktabs}
\usepackage{graphicx}
\usepackage{array}
\usepackage{multirow}

\title{分块五对角矩阵:二维热传导方程数值离散结构解析}
\author{数值传热学分析}
\date{\today}

\begin{document}

\maketitle

\section{引言}
在二维热传导方程的数值求解中,隐式向后差分格式会产生特定的矩阵结构。本文深入分析这种结构如何从五对角矩阵演化为分块五对角矩阵,揭示其数学本质及物理意义。

\section{五对角矩阵基本结构}
考虑$N_x \times N_y$网格系统,其线性系统矩阵为五对角形式:
$$
A = \begin{pmatrix}
d_0 & e_0 & f_0 & 0 & \cdots & \cdots & 0 \\
c_1 & d_1 & e_1 & f_1 & \ddots & & \vdots \\
b_2 & c_2 & d_2 & e_2 & f_2 & \ddots & \vdots \\
0 & b_3 & c_3 & d_3 & \ddots & \ddots & 0 \\
\vdots & \ddots & \ddots & \ddots & \ddots & e_{n-2} & f_{n-2} \\
\vdots & & \ddots & b_{n-1} & c_{n-1} & d_{n-1} & e_{n-1} \\
0 & \cdots & \cdots & 0 & b_n & c_n & d_n
\end{pmatrix}
$$
其中非零元素出现在主对角线和其相邻的四个对角线上。

\section{二维热传导问题离散方程}
二维热传导方程隐式离散格式:
\begin{equation}
-r_x T_{i-1,j}^{n+1} - r_y T_{i,j-1}^{n+1} + (1+2r_x+2r_y)T_{i,j}^{n+1} - r_x T_{i+1,j}^{n+1} - r_y T_{i,j+1}^{n+1} = T_{i,j}^n
\label{eq:discrete}
\end{equation}
其中$r_x = \frac{\alpha \Delta t}{(\Delta x)^2}$, $r_y = \frac{\alpha \Delta t}{(\Delta y)^2}$。

\section{分块矩阵结构推导}

\subsection{节点排序方法}
采用行优先(lexicographical)排序:
$$
\mathbf{T} = \begin{pmatrix}
T_{0,0} & T_{0,1} & \cdots & T_{0,N_y-1} & T_{1,0} & \cdots & T_{N_x-1,N_y-1}
\end{pmatrix}^T
$$
索引映射:$k(i,j) = i \cdot N_y + j$

\subsection{矩阵元素分析}
\begin{table}[htbp]
\centering
\caption{矩阵元素与物理意义对应关系}
\begin{tabular}{c|c|c}
类型 & 矩阵位置 & 物理耦合 \\ \hline
自耦合 & $A_{k,k} = 1+2r_x+2r_y$ & 节点自身 \\
行内左邻 & $A_{k,k-1} = -r_y$ & $T_{i,j} \leftrightarrow T_{i,j-1}$ \\
行内右邻 & $A_{k,k+1} = -r_y$ & $T_{i,j} \leftrightarrow T_{i,j+1}$ \\
上行同列 & $A_{k,k-N_y} = -r_x$ & $T_{i,j} \leftrightarrow T_{i-1,j}$ \\
下行同列 & $A_{k,k+N_y} = -r_x$ & $T_{i,j} \leftrightarrow T_{i+1,j}$ \\
\end{tabular}
\end{table}

\subsection{分块矩阵形成}
将矩阵划分为$N_x \times N_x$个块,每个块大小为$N_y \times N_y$:
$$
A = \begin{pmatrix}
B_0 & C_0 & 0 & \cdots & 0 \\
A_1 & B_1 & C_1 & \ddots & \vdots \\
0 & A_2 & B_2 & \ddots & 0 \\
\vdots & \ddots & \ddots & \ddots & C_{N_x-2} \\
0 & \cdots & 0 & A_{N_x-1} & B_{N_x-1}
\end{pmatrix}
$$

\subsubsection{对角块$B_i$结构}
$B_i$描述第$i$行内的耦合关系,为三对角矩阵:
$$
B_i = \begin{pmatrix}
1+2r_x+2r_y & -r_y & 0 & \cdots & 0 \\
-r_y & 1+2r_x+2r_y & -r_y & \ddots & \vdots \\
0 & -r_y & 1+2r_x+2r_y & \ddots & 0 \\
\vdots & \ddots & \ddots & \ddots & -r_y \\
0 & \cdots & 0 & -r_y & 1+2r_x+2r_y
\end{pmatrix}
$$

\subsubsection{次对角块$A_i$结构}
$A_i$描述与上一行$(i-1)$的耦合,为对角矩阵:
$$
A_i = \begin{pmatrix}
-r_x & 0 & \cdots & 0 \\
0 & -r_x & \ddots & \vdots \\
\vdots & \ddots & \ddots & 0 \\
0 & \cdots & 0 & -r_x
\end{pmatrix}
$$

\subsubsection{上次对角块$C_i$结构}
$C_i$描述与下一行$(i+1)$的耦合,也为对角矩阵:
$$
C_i = \begin{pmatrix}
-r_x & 0 & \cdots & 0 \\
0 & -r_x & \ddots & \vdots \\
\vdots & \ddots & \ddots & 0 \\
0 & \cdots & 0 & -r_x
\end{pmatrix}
$$

\section{矩阵性质分析}

\subsection{对角占优性}
对角块$B_i$严格对角占优:
$$
|B_i[j,j]| = |1+2r_x+2r_y| > 2r_x + 2r_y = \sum_{k \neq j} |B_i[j,k]| + |A_i[j,j]| + |C_i[j,j]|
$$

\subsection{稀疏性分析}
非零元素比例:
$$
\frac{\text{非零元素数}}{\text{总元素数}} = \frac{5N_xN_y}{(N_xN_y)^2} = \frac{5}{N_xN_y}
$$
例如当$N_x = N_y = 100$时,非零元素占比仅0.05\%。

\section{边界条件处理}

\subsection{第二类边界条件离散}
以左边界($i=0$)为例:
$$
-k \frac{\partial T}{\partial x}\bigg|_{x=0} \approx -k \frac{-3T_{0,j} + 4T_{1,j} - T_{2,j}}{2\Delta x} = q_{\text{left}}
$$

\subsection{角点处理}
左下角($i=0,j=0$)采用混合近似:
$$
\begin{cases}
-k \frac{-3T_{0,0} + 4T_{1,0} - T_{2,0}}{2\Delta x} = q_x \\
-k \frac{-3T_{0,0} + 4T_{0,1} - T_{0,2}}{2\Delta y} = q_y
\end{cases}
$$

\section{分块矩阵求解优势}

\subsection{存储优化}
传统存储:$O((N_xN_y)^2)$\\
分块存储:$O(5N_xN_y)$

\subsection{求解算法}
块三对角系统求解算法:
\begin{align*}
&\text{// 前向消去}\\
&\text{for } i = 1 \text{ to } N_x-1 \\
&\quad B_i := B_i - A_i \cdot B_{i-1}^{-1} \cdot C_{i-1} \\
&\quad \mathbf{b}_i := \mathbf{b}_i - A_i \cdot B_{i-1}^{-1} \cdot \mathbf{b}_{i-1} \\
& \\
&\text{// 回代}\\
&\mathbf{T}_{N_x-1} = B_{N_x-1}^{-1} \mathbf{b}_{N_x-1} \\
&\text{for } i = N_x-2 \text{ downto } 0 \\
&\quad \mathbf{T}_i = B_i^{-1} (\mathbf{b}_i - C_i \mathbf{T}_{i+1})
\end{align*}

\section{结论}
本文证明二维热传导方程隐式离散:
\begin{itemize}
\item 自然形成分块五对角矩阵结构
\item 分块结构是物理耦合关系的数学表征
\item 提供高效存储和求解框架
\item 为大型问题求解提供理论基础
\end{itemize}

分块结构不是人为选择,而是热传导物理特性在数值离散中的必然呈现。

\end{document}