\documentclass{ctexart}
\usepackage{amsmath}
\usepackage{amssymb}
\usepackage{geometry}
\geometry{a4paper, margin=1in}
\usepackage{booktabs}
\usepackage{graphicx}
\usepackage{hyperref}

\title{二维非稳态热传导方程隐式差分格式矩阵结构分析}
\author{传热学数值分析}
\date{\today}

\begin{document}

\maketitle

\section{控制方程}
考虑二维非稳态热传导方程:
\begin{equation}
\frac{\partial T}{\partial t} = \alpha \left( \frac{\partial^2 T}{\partial x^2} + \frac{\partial^2 T}{\partial y^2} \right)
\label{eq:heat}
\end{equation}
其中:
\begin{itemize}
\item $T = T(x,y,t)$ 为温度场
\item $\alpha$ 为热扩散系数
\item $t$ 为时间变量
\item $(x,y)$ 为空间坐标
\end{itemize}

\section{数值离散}

\subsection{空间-时间离散}
将求解域离散为均匀网格:
\begin{itemize}
\item 空间步长:$\Delta x$, $\Delta y$
\item 时间步长:$\Delta t$
\item 网格点索引:$(i,j)$ 对应 $(x_i,y_j)$
\item 时间步索引:$n$ 对应 $t_n$
\end{itemize}

\subsection{隐式向后差分格式}
对时间项采用一阶向后差分:
\begin{equation}
\frac{\partial T}{\partial t} \bigg|_{i,j}^{n+1} \approx \frac{T_{i,j}^{n+1} - T_{i,j}^n}{\Delta t}
\end{equation}

对空间项采用二阶中心差分:
\begin{align}
\frac{\partial^2 T}{\partial x^2} \bigg|_{i,j}^{n+1} &\approx \frac{T_{i-1,j}^{n+1} - 2T_{i,j}^{n+1} + T_{i+1,j}^{n+1}}{(\Delta x)^2} \\
\frac{\partial^2 T}{\partial y^2} \bigg|_{i,j}^{n+1} &\approx \frac{T_{i,j-1}^{n+1} - 2T_{i,j}^{n+1} + T_{i,j+1}^{n+1}}{(\Delta y)^2}
\end{align}

\subsection{全离散方程}
代入控制方程(\ref{eq:heat})得:
\begin{equation}
\frac{T_{i,j}^{n+1} - T_{i,j}^n}{\Delta t} = \alpha \left( \frac{T_{i-1,j}^{n+1} - 2T_{i,j}^{n+1} + T_{i+1,j}^{n+1}}{(\Delta x)^2} + \frac{T_{i,j-1}^{n+1} - 2T_{i,j}^{n+1} + T_{i,j+1}^{n+1}}{(\Delta y)^2} \right)
\end{equation}

\section{矩阵系统推导}

\subsection{方程重排}
定义常数:
\begin{align}
r_x &= \frac{\alpha \Delta t}{(\Delta x)^2} \\
r_y &= \frac{\alpha \Delta t}{(\Delta y)^2}
\end{align}

重排方程:
\begin{equation}
-r_x T_{i-1,j}^{n+1} - r_y T_{i,j-1}^{n+1} + (1 + 2r_x + 2r_y) T_{i,j}^{n+1} - r_x T_{i+1,j}^{n+1} - r_y T_{i,j+1}^{n+1} = T_{i,j}^n
\label{eq:discrete}
\end{equation}

\subsection{节点排序与索引映射}
将二维网格按行优先顺序展开为一维向量:
\begin{equation}
\mathbf{T}^{n+1} = \begin{bmatrix}
T_{0,0} \\
T_{0,1} \\
\vdots \\
T_{0,N_y-1} \\
T_{1,0} \\
T_{1,1} \\
\vdots \\
T_{N_x-1,N_y-1}
\end{bmatrix}
\end{equation}
索引映射:
\begin{equation}
k(i,j) = i \times N_y + j
\end{equation}
其中 $N_y$ 为 y 方向节点数。

\subsection{矩阵元素分析}
根据方程(\ref{eq:discrete}),对内部节点 $(i,j)$:
\begin{center}
\begin{tabular}{c|c|l}
位置 & 矩阵行 & 非零元素 \\
\midrule
自身 & $k(i,j)$ & $A_{k,k} = 1 + 2r_x + 2r_y$ \\
左邻 & $k(i-1,j)$ & $A_{k,k-N_y} = -r_x$ \\
右邻 & $k(i+1,j)$ & $A_{k,k+N_y} = -r_x$ \\
下邻 & $k(i,j-1)$ & $A_{k,k-1} = -r_y$ \\
上邻 & $k(i,j+1)$ & $A_{k,k+1} = -r_y$ \\
\end{tabular}
\end{center}

\subsection{矩阵结构}
系统矩阵 $A$ 为分块五对角矩阵:
\begin{equation}
A = \begin{bmatrix}
B_0 & C_0 & 0 & \cdots & 0 \\
A_1 & B_1 & C_1 & \ddots & \vdots \\
0 & A_2 & B_2 & \ddots & 0 \\
\vdots & \ddots & \ddots & \ddots & C_{N_x-2} \\
0 & \cdots & 0 & A_{N_x-1} & B_{N_x-1}
\end{bmatrix}
\end{equation}

其中:
\begin{itemize}
\item $B_i$:大小为 $N_y \times N_y$ 的三对角矩阵
\item $A_i$, $C_i$:大小为 $N_y \times N_y$ 的对角矩阵
\end{itemize}

具体子矩阵结构:
\begin{align}
B_i &= \begin{bmatrix}
d & -r_y & 0 & \cdots & 0 \\
-r_y & d & -r_y & \ddots & \vdots \\
0 & -r_y & d & \ddots & 0 \\
\vdots & \ddots & \ddots & \ddots & -r_y \\
0 & \cdots & 0 & -r_y & d
\end{bmatrix}, \quad d = 1 + 2r_x + 2r_y \\
A_i &= \operatorname{diag}(-r_x, -r_x, \dots, -r_x) \\
C_i &= \operatorname{diag}(-r_x, -r_x, \dots, -r_x)
\end{align}

\section{物理与数学特性}

\subsection{物理意义}
矩阵结构反映热传导的物理特性:
\begin{itemize}
\item \textbf{局部性}:热传导仅与相邻节点耦合
\item \textbf{各向异性}:$r_x \neq r_y$ 时,x 和 y 方向传热不同
\item \textbf{守恒性}:对角占优保证数值稳定性
\end{itemize}

\subsection{数学特性}
\begin{itemize}
\item \textbf{稀疏性}:每行仅 5 个非零元素
\item \textbf{对称性}:当 $\Delta x = \Delta y$ 时对称
\item \textbf{对角占优}:$|A_{k,k}| \geq \sum_{j \neq k} |A_{k,j}|$
\item \textbf{正定性}:特征值实部为正
\end{itemize}

\section{求解方法}

\subsection{直接解法}
\begin{equation}
A\mathbf{T}^{n+1} = \mathbf{b}^n
\end{equation}
其中 $\mathbf{b}^n$ 包含右端项和边界条件。

常用算法:
\begin{itemize}
\item LU 分解(适用于中小规模问题)
\item 追赶法(特殊五对角矩阵)
\item 稀疏矩阵求解器(如 SuperLU)
\end{itemize}

\subsection{迭代解法}
\begin{itemize}
\item Gauss-Seidel 迭代
\item 共轭梯度法(CG)
\item 多重网格法(高效)
\end{itemize}

\subsection{交替方向隐式法(ADI)}
将二维问题分解为两个一维问题:
\begin{align}
\left(1 - r_x \delta_x^2\right) T^{n+1/2} &= \left(1 + r_y \delta_y^2\right) T^n \\
\left(1 - r_y \delta_y^2\right) T^{n+1} &= \left(1 + r_x \delta_x^2\right) T^{n+1/2}
\end{align}
每步仅需求解三对角矩阵系统。

\section{边界条件处理}

\subsection{第二类边界条件}
对边界节点 $(i,j) \in \partial \Omega$:
\begin{equation}
-k \frac{\partial T}{\partial n} = q
\end{equation}
离散形式:
\begin{itemize}
\item \textbf{左边界}($i=0$):$-k \frac{T_{1,j} - T_{0,j}}{\Delta x} \approx q_{\text{left}}$
\item \textbf{右边界}($i=N_x-1$):$-k \frac{T_{N_x-1,j} - T_{N_x-2,j}}{\Delta x} \approx q_{\text{right}}$
\item 类似处理其他边界
\end{itemize}

\subsection{角点处理}
如左下角 $(0,0)$:
\begin{equation}
\frac{-k}{\sqrt{2}} \left( \frac{T_{1,0} - T_{0,0}}{\Delta x} + \frac{T_{0,1} - T_{0,0}}{\Delta y} \right) \approx q_{\text{corner}}
\end{equation}

\section{稳定性分析}

隐式格式的稳定性由矩阵特征值决定:
\begin{equation}
\rho(I - \Delta t \cdot A) < 1
\end{equation}
其中 $\rho$ 为谱半径。

von Neumann 稳定性分析显示隐式格式\textbf{无条件稳定}:
\begin{equation}
|g(k)| = \frac{1}{|1 + 4r_x \sin^2(k_x \Delta x / 2) + 4r_y \sin^2(k_y \Delta y / 2)|} \leq 1
\end{equation}
其中 $g(k)$ 为增长因子。

\section{总结}
二维热传导方程隐式离散化形成五对角矩阵,由离散方程的局部耦合性质决定:
\begin{enumerate}
\item 热传导的物理机制仅涉及相邻节点
\item 空间二阶导数离散产生五点耦合
\item 行优先排序产生五对角块结构
\end{enumerate}
此结构虽然比一维复杂,但仍保持稀疏性,可高效求解。

\end{document}